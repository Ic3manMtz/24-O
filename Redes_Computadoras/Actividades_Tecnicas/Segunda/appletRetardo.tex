%-----------Applet de retardo de transmisión y propagación 

\section*{Applet de retardo de transmisi\'on y propagaci\'on}

	\begin{description}
		\item[¿Bajo qué circunstancias el inicio del paquete alcanza al receptor antes de que su transmisi\'on se haya completado?] 
		Esto pasa cuando el retardo de transmisión es mayor al retardo de propagación.
		
		\item[¿Qu\'e relaci\'on entre $d_{trans}$ y $d_{prop}$ permitir\'ia que el paquete ocupara enteramente el enlace? Es decir, que el primer bit del paquete se encuentre justo llegando al otro extremo del enlace y que el \'ultimo bit haya apenas abandonado el transmisor]
		Esto sería posible cuando el retardo de transmisión es igual al retardo de propagación
		
		\item[Exlpique el concepto del producto del retardo por el anchi de banda $(R\;x\;d_{prop})$ utilizando este applet. Para la configuraci\'on que eligi\'o en el inciso anterior, ¿cu\'al es el ancho de cada bit en metros?]
		El retardo por el ancho de bando se podría explicar como la cantidad de bits que puede haber en el enlace a la vez. La configuración que usé para el ejercicio anterior es de 10Km, 10Mbps, 100Bytes, lo cual no es dtrans = dprop , pero es uno de los mejores aproximados para conseguirlo, entonces al ser (10Km/2.8x108)*(100Mbps) = 3,571.43 bits
		Por lo tanto, para calcular el tamaño de bit por metro sería: 
		d/R*dprop = 10km / 3,571.43 = 2.8m/bit aprox
		
	\end{description}