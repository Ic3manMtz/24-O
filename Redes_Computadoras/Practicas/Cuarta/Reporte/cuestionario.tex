\section*{Cuestionario}

\subsection*{Describa qu\'e hace cada script ejecutado en la pr\'actica, ilustrando las partes en donde se configura cada una de las
capas del Modelo TCP/IP.}

\subsubsection*{Script 1}
\noindent Ya que este script no incluye nodos, enlaces, o tr\'afico de paquetes, vamos a describir cada parte del script.

\noindent Crea una instancia del simulador NS-2 y la asigna a la variable ns. Este objeto es el n\'ucleo de la simulaci\'on, permitiendo
condigurar nodos, tr\'afico, y otros elementos.
\begin{figure}[H]
  \centering
  \begin{lstlisting}[frame=single, breaklines=true, basicstyle=\footnotesize\ttfamily, breakatwhitespace=false, 
    columns=flexible, tabsize=2, showstringspaces=false]
    set ns [new Simulator]
  \end{lstlisting}
\end{figure}

Crea un archivo \textbf{out.nam} en modo escritura. Este archivo registrar\'a todos los eventos de simulaci\'on, como el env\'io/recepci\'on
de paquetes y otros eventos relevantes
\begin{figure}[H]
  \centering
  \begin{lstlisting}[frame=single, breaklines=true, basicstyle=\footnotesize\ttfamily, breakatwhitespace=false, 
    columns=flexible, tabsize=2, showstringspaces=false]
    set nf [open out.nam w]
  \end{lstlisting}
\end{figure}

Le indica al simulador que registre todos los eventos de simulaci\'on en el archivo \textbf{out.nam} previamente creado.
\begin{figure}[H]
  \centering
  \begin{lstlisting}[frame=single, breaklines=true, basicstyle=\footnotesize\ttfamily, breakatwhitespace=false, 
    columns=flexible, tabsize=2, showstringspaces=false]
    $ns namtrace-all $nf
  \end{lstlisting}
\end{figure}

\newpage

Define una funci\'on llamada \textbf{finish} que:
\begin{description}
  \item[global ns nf] : Declara las variables \textbf{ns} y \textbf{nf} como globales para usarlas dentro de la funci\'on.
  \item[\$ns flush-trace] : Asegura que toda la informaci\'on de traza se escriba en el archivo antes de cerrarlo.
  \item[close \$nf] : Cierra el archivo de traza \textbf{out.nam}
  \item[exec nam out.nam \&] : Ejecuta NAM y abre el archivo \textbf{out.nam} para visualizar la simulaci\'on.
  \item[exit 0] : Finaliza la ejecuci\'on del script de manera exitosa.      
\end{description}
\begin{figure}[H]
  \centering
  \begin{lstlisting}[frame=single, breaklines=true, basicstyle=\footnotesize\ttfamily, breakatwhitespace=false, 
    columns=flexible, tabsize=2, showstringspaces=false]
    proc finish {} {
      global ns nf
      $ns flush-trace
      close $nf
      exec nam out.nam &
      exit 0
    }
  \end{lstlisting}
\end{figure}

Programa la ejecuci\'on del procedimiento \textbf{finish} despu\'es de 5 segundos de tiempo simulado. Esto significa, que 
independientemente de la configuraci\'on adicional, la simulaci\'on terminar\'a a los 5 segundos.
\begin{figure}[H]
  \centering
  \begin{lstlisting}[frame=single, breaklines=true, basicstyle=\footnotesize\ttfamily, breakatwhitespace=false, 
    columns=flexible, tabsize=2, showstringspaces=false]
    $ns at 5.0 "finish"
  \end{lstlisting}
\end{figure}

Inicia la ejecuci\'on de la simulaci\'on. A partir de este punto, todos los eventos configurados en el simulador 
se procesan secuencialmente.
\begin{figure}[H]
  \centering
  \begin{lstlisting}[frame=single, breaklines=true, basicstyle=\footnotesize\ttfamily, breakatwhitespace=false, 
    columns=flexible, tabsize=2, showstringspaces=false]
    $ns run
  \end{lstlisting}
\end{figure}

\newpage

\subsubsection*{Script 2}

\noindent Como vimos en el punto anterior las siguientes l\'ineas del script son parte de la configuraci\'on general y la
creaci\'on del procedimiento de finalizaci\'on.
\begin{figure}[H]
  \centering
  \begin{lstlisting}[frame=single, breaklines=true, basicstyle=\footnotesize\ttfamily, breakatwhitespace=false, 
    columns=flexible, tabsize=2, showstringspaces=false]
    set ns [new Simulator]
    set nf [open out.nam w]
    $ns namtrace-all $nf
    proc finish {} {
      global ns nf
      $ns flush-trace
      close $nf
      exec nam out.nam &
      exit 0
    }
  \end{lstlisting}
\end{figure}

\noindent Crea dos nodos (\textit{n0 y n1}) que actuar\'an como origen y destino en la simulaci\'on. Esta parte del script corresponde a la
\textbf{Capa de Enlace de Datos}, ya que los nodos representan interfaces de red.
\begin{figure}[H]
  \centering
  \begin{lstlisting}[frame=single, breaklines=true, basicstyle=\footnotesize\ttfamily, breakatwhitespace=false, 
    columns=flexible, tabsize=2, showstringspaces=false]
    set n0 [$ns node]
    set n1 [$ns node]
  \end{lstlisting}
\end{figure}

\noindent Crea un enlace d\'uplex entre los nodos \textbf{n0} y \textbf{n1} con:
\begin{description}
  \item[Ancho de banda] : 1 Mbps
  \item[Retardo] : 10 ms
  \item[Cola] : DropTail (cola FIFO)   
\end{description}
La relaci\'on con el modelo TCP/IP es con la \textbf{Capa de Enlace de Datos} ya que representa el enlace f\'isico y la cola de paquetes.
As\'i como la \textbf{Capa de Red} ya que el retardo y el ancho de banda afectan la entrega de paquetes.
\begin{figure}[H]
  \centering
  \begin{lstlisting}[frame=single, breaklines=true, basicstyle=\footnotesize\ttfamily, breakatwhitespace=false, 
    columns=flexible, tabsize=2, showstringspaces=false]
    $ns duplex-link $n0 $n1 1Mb 10ms DropTail
  \end{lstlisting}
\end{figure}

\noindent Crea un agente UDP (\textit{udp0}) y lo asocia al nodo \textbf{n0}. Se relaciona con el modelo TCP/IP ya que representa la \textbf{Capa de Transporte}, 
ya que el protocolo UDP es un protocolo en esta capa.
\begin{figure}[H]
  \centering
  \begin{lstlisting}[frame=single, breaklines=true, basicstyle=\footnotesize\ttfamily, breakatwhitespace=false, 
    columns=flexible, tabsize=2, showstringspaces=false]
    set udp0 [new Agent/UDP]
    $ns attach-agent $n0 $udp0    
  \end{lstlisting}
\end{figure}

\newpage

\noindent Crea una  fuente de tr\'afico CBR (\textit{cbr0}) con:
\begin{description}
  \item[Tamaño de paquete] : 1000 bytes
  \item[Intervalo entre paquetes] : 0.005 segundos
  \item[Se conecta al agente UDP \textit{udp0}]  
\end{description}
Representa la \textbf{Capa de Aplicaci\'on}, ya que genera tr\'afico constante para enviar a trav\'es de la red.
\begin{figure}[H]
  \begin{lstlisting}[frame=single, breaklines=true, basicstyle=\footnotesize\ttfamily, breakatwhitespace=false, 
    columns=flexible, tabsize=2, showstringspaces=false]
    set cbr0 [new Application/Traffic/CBR]
    $cbr0 set packetSize_ 1000
    $cbr0 set interval_ 0.005
    $cbr0 attach-agent $udp0
  \end{lstlisting}
\end{figure}

\noindent Crea un agente Null (\textit{null0}) que act\'ua como receptor de tr\'afico en el nodo \textbf{n1}. Representa la \textbf{Capa de Transporte}, ya
que recibe y descarta paquetes sin procesarlos.
\begin{figure}[H]
  \begin{lstlisting}[frame=single, breaklines=true, basicstyle=\footnotesize\ttfamily, breakatwhitespace=false, 
    columns=flexible, tabsize=2, showstringspaces=false]
    set null0 [new Agent/Null]
    $ns attach-agent $n1 $null0    
  \end{lstlisting}
\end{figure}

\noindent Conecta el agente UDP (\textit{udp0}) en el nodo \textbf{n0} con el agente Null (\textit{null0}) en el nodo \textbf{n1}. Relaciona las 
\textbf{Capas de Transporte y Red}, al definir una ruta l\'ogica entre el origen y el destino.
\begin{figure}[H]
  \begin{lstlisting}[frame=single, breaklines=true, basicstyle=\footnotesize\ttfamily, breakatwhitespace=false, 
    columns=flexible, tabsize=2, showstringspaces=false]
    $ns connect $udp0 $null0
  \end{lstlisting}
\end{figure}

\noindent Programa el inicio del tr\'afico CBR a los 0.5 segundos y su detenci\'on a los 4.5 segundos de tiempo de simulaci\'on. Impacta la
\textbf{Capa de Aplicaci\'on}, ya que controla cu\'ando se env\'ian los datos.
\begin{figure}[H]
  \begin{lstlisting}[frame=single, breaklines=true, basicstyle=\footnotesize\ttfamily, breakatwhitespace=false, 
    columns=flexible, tabsize=2, showstringspaces=false]
    $ns at 0.5 "$cbr0 start"
    $ns at 4.5 "$cbr0 stop"    
  \end{lstlisting}
\end{figure}

\begin{enumerate}
  \item Capa de Enlace de Datos
  \begin{itemize}
    \item Creaci\'on de nodos.
    \item Configuraci\'on del enlace d\'uplex.
  \end{itemize}

  \item Capa de Red
  \begin{itemize}
    \item Configuraci\'on del enlace afecta el enrutamiento y entrega de paquetes.
  \end{itemize}

  \item Capa de Transporte
  \begin{itemize}
    \item Configuraci\'on del agente UDP y receptor Null.
  \end{itemize}

  \item Capa de Aplicaci\'on
  \begin{itemize}
    \item Generaci\'on de tr\'afico constante (\textit{CBR}).
    \item Programaci\'on del inicio y finalizaci\'on del tr\'afico.
  \end{itemize}
\end{enumerate}

\subsubsection*{Script 3}

\subsection*{¿C\'omo podr\'ia usarse NS-2 para explicar la relaci\'on entre el retardo de propagaci\'on y el de transmisi\'on en un
enlace?¿Qu\'e parte de un script TCL deber\'ia modifical para ejemplificar esto?}

\subsection*{Para el segundo ejemplo de la pr\'actica, modifique el c\'odigo para que el tamaño del paquete sea igual a 1,500 bytes.
¿Qu\'e diferencias observa en la animaci\'on con NAM?}

\subsection*{Investigue la disponibilidad de simuladores de red y realice una tabla con al menos cinco simuladores que compare sus
caracter\'isticas b\'asicas.}
