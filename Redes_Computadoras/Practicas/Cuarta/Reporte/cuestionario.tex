\section*{Cuestionario}

\subsection*{Describa qu\'e hace cada script ejecutado en la pr\'actica, ilustrando las partes en donde se configura cada una de las
capas del Modelo TCP/IP.}

\subsubsection*{Script 1}
\noindent Ya que este script no incluye nodos, enlaces, o tr\'afico de paquetes, vamos a describir cada parte del script.

\noindent Crea una instancia del simulador NS-2 y la asigna a la variable ns. Este objeto es el n\'ucleo de la simulaci\'on, permitiendo
condigurar nodos, tr\'afico, y otros elementos.
\begin{figure}[H]
  \centering
  \begin{lstlisting}[frame=single, breaklines=true, basicstyle=\footnotesize\ttfamily, breakatwhitespace=false, 
    columns=flexible, tabsize=2, showstringspaces=false]
    set ns [new Simulator]
  \end{lstlisting}
\end{figure}

Crea un archivo \textbf{out.nam} en modo escritura. Este archivo registrar\'a todos los eventos de simulaci\'on, como el env\'io/recepci\'on
de paquetes y otros eventos relevantes
\begin{figure}[H]
  \centering
  \begin{lstlisting}[frame=single, breaklines=true, basicstyle=\footnotesize\ttfamily, breakatwhitespace=false, 
    columns=flexible, tabsize=2, showstringspaces=false]
    set nf [open out.nam w]
  \end{lstlisting}
\end{figure}

Le indica al simulador que registre todos los eventos de simulaci\'on en el archivo \textbf{out.nam} previamente creado.
\begin{figure}[H]
  \centering
  \begin{lstlisting}[frame=single, breaklines=true, basicstyle=\footnotesize\ttfamily, breakatwhitespace=false, 
    columns=flexible, tabsize=2, showstringspaces=false]
    $ns namtrace-all $nf
  \end{lstlisting}
\end{figure}

\newpage

Define una funci\'on llamada \textbf{finish} que:
\begin{description}
  \item[global ns nf] : Declara las variables \textbf{ns} y \textbf{nf} como globales para usarlas dentro de la funci\'on.
  \item[\$ns flush-trace] : Asegura que toda la informaci\'on de traza se escriba en el archivo antes de cerrarlo.
  \item[close \$nf] : Cierra el archivo de traza \textbf{out.nam}
  \item[exec nam out.nam \&] : Ejecuta NAM y abre el archivo \textbf{out.nam} para visualizar la simulaci\'on.
  \item[exit 0] : Finaliza la ejecuci\'on del script de manera exitosa.      
\end{description}
\begin{figure}[H]
  \centering
  \begin{lstlisting}[frame=single, breaklines=true, basicstyle=\footnotesize\ttfamily, breakatwhitespace=false, 
    columns=flexible, tabsize=2, showstringspaces=false]
    proc finish {} {
      global ns nf
      $ns flush-trace
      close $nf
      exec nam out.nam &
      exit 0
    }
  \end{lstlisting}
\end{figure}

Programa la ejecuci\'on del procedimiento \textbf{finish} despu\'es de 5 segundos de tiempo simulado. Esto significa, que 
independientemente de la configuraci\'on adicional, la simulaci\'on terminar\'a a los 5 segundos.
\begin{figure}[H]
  \centering
  \begin{lstlisting}[frame=single, breaklines=true, basicstyle=\footnotesize\ttfamily, breakatwhitespace=false, 
    columns=flexible, tabsize=2, showstringspaces=false]
    $ns at 5.0 "finish"
  \end{lstlisting}
\end{figure}

Inicia la ejecuci\'on de la simulaci\'on. A partir de este punto, todos los eventos configurados en el simulador 
se procesan secuencialmente.
\begin{figure}[H]
  \centering
  \begin{lstlisting}[frame=single, breaklines=true, basicstyle=\footnotesize\ttfamily, breakatwhitespace=false, 
    columns=flexible, tabsize=2, showstringspaces=false]
    $ns run
  \end{lstlisting}
\end{figure}

\newpage

\subsubsection*{Script 2}

\noindent Como vimos en el punto anterior las siguientes l\'ineas del script son parte de la configuraci\'on general y la
creaci\'on del procedimiento de finalizaci\'on.
\begin{figure}[H]
  \centering
  \begin{lstlisting}[frame=single, breaklines=true, basicstyle=\footnotesize\ttfamily, breakatwhitespace=false, 
    columns=flexible, tabsize=2, showstringspaces=false]
    set ns [new Simulator]
    set nf [open out.nam w]
    $ns namtrace-all $nf
    proc finish {} {
      global ns nf
      $ns flush-trace
      close $nf
      exec nam out.nam &
      exit 0
    }
  \end{lstlisting}
\end{figure}

\noindent Crea dos nodos (\textit{n0 y n1}) que actuar\'an como origen y destino en la simulaci\'on. Esta parte del script corresponde a la
\textbf{Capa de Enlace de Datos}, ya que los nodos representan interfaces de red.
\begin{figure}[H]
  \centering
  \begin{lstlisting}[frame=single, breaklines=true, basicstyle=\footnotesize\ttfamily, breakatwhitespace=false, 
    columns=flexible, tabsize=2, showstringspaces=false]
    set n0 [$ns node]
    set n1 [$ns node]
  \end{lstlisting}
\end{figure}

\noindent Crea un enlace d\'uplex entre los nodos \textbf{n0} y \textbf{n1} con:
\begin{description}
  \item[Ancho de banda] : 1 Mbps
  \item[Retardo] : 10 ms
  \item[Cola] : DropTail (cola FIFO)   
\end{description}
La relaci\'on con el modelo TCP/IP es con la \textbf{Capa de Enlace de Datos} ya que representa el enlace f\'isico y la cola de paquetes.
As\'i como la \textbf{Capa de Red} ya que el retardo y el ancho de banda afectan la entrega de paquetes.
\begin{figure}[H]
  \centering
  \begin{lstlisting}[frame=single, breaklines=true, basicstyle=\footnotesize\ttfamily, breakatwhitespace=false, 
    columns=flexible, tabsize=2, showstringspaces=false]
    $ns duplex-link $n0 $n1 1Mb 10ms DropTail
  \end{lstlisting}
\end{figure}

\noindent Crea un agente UDP (\textit{udp0}) y lo asocia al nodo \textbf{n0}. Se relaciona con el modelo TCP/IP ya que representa la \textbf{Capa de Transporte}, 
ya que el protocolo UDP es un protocolo en esta capa.
\begin{figure}[H]
  \centering
  \begin{lstlisting}[frame=single, breaklines=true, basicstyle=\footnotesize\ttfamily, breakatwhitespace=false, 
    columns=flexible, tabsize=2, showstringspaces=false]
    set udp0 [new Agent/UDP]
    $ns attach-agent $n0 $udp0    
  \end{lstlisting}
\end{figure}

\newpage

\noindent Crea una  fuente de tr\'afico CBR (\textit{cbr0}) con:
\begin{description}
  \item[Tamaño de paquete] : 1000 bytes
  \item[Intervalo entre paquetes] : 0.005 segundos
  \item[Se conecta al agente UDP \textit{udp0}]  
\end{description}
Representa la \textbf{Capa de Aplicaci\'on}, ya que genera tr\'afico constante para enviar a trav\'es de la red.
\begin{figure}[H]
  \begin{lstlisting}[frame=single, breaklines=true, basicstyle=\footnotesize\ttfamily, breakatwhitespace=false, 
    columns=flexible, tabsize=2, showstringspaces=false]
    set cbr0 [new Application/Traffic/CBR]
    $cbr0 set packetSize_ 1000
    $cbr0 set interval_ 0.005
    $cbr0 attach-agent $udp0
  \end{lstlisting}
\end{figure}

\noindent Crea un agente Null (\textit{null0}) que act\'ua como receptor de tr\'afico en el nodo \textbf{n1}. Representa la \textbf{Capa de Transporte}, ya
que recibe y descarta paquetes sin procesarlos.
\begin{figure}[H]
  \begin{lstlisting}[frame=single, breaklines=true, basicstyle=\footnotesize\ttfamily, breakatwhitespace=false, 
    columns=flexible, tabsize=2, showstringspaces=false]
    set null0 [new Agent/Null]
    $ns attach-agent $n1 $null0    
  \end{lstlisting}
\end{figure}

\noindent Conecta el agente UDP (\textit{udp0}) en el nodo \textbf{n0} con el agente Null (\textit{null0}) en el nodo \textbf{n1}. Relaciona las 
\textbf{Capas de Transporte y Red}, al definir una ruta l\'ogica entre el origen y el destino.
\begin{figure}[H]
  \begin{lstlisting}[frame=single, breaklines=true, basicstyle=\footnotesize\ttfamily, breakatwhitespace=false, 
    columns=flexible, tabsize=2, showstringspaces=false]
    $ns connect $udp0 $null0
  \end{lstlisting}
\end{figure}

\noindent Programa el inicio del tr\'afico CBR a los 0.5 segundos y su detenci\'on a los 4.5 segundos de tiempo de simulaci\'on. Impacta la
\textbf{Capa de Aplicaci\'on}, ya que controla cu\'ando se env\'ian los datos.
\begin{figure}[H]
  \begin{lstlisting}[frame=single, breaklines=true, basicstyle=\footnotesize\ttfamily, breakatwhitespace=false, 
    columns=flexible, tabsize=2, showstringspaces=false]
    $ns at 0.5 "$cbr0 start"
    $ns at 4.5 "$cbr0 stop"    
  \end{lstlisting}
\end{figure}

\begin{enumerate}
  \item Capa de Enlace de Datos
  \begin{itemize}
    \item Creaci\'on de nodos.
    \item Configuraci\'on del enlace d\'uplex.
  \end{itemize}

  \item Capa de Red
  \begin{itemize}
    \item Configuraci\'on del enlace afecta el enrutamiento y entrega de paquetes.
  \end{itemize}

  \item Capa de Transporte
  \begin{itemize}
    \item Configuraci\'on del agente UDP y receptor Null.
  \end{itemize}

  \item Capa de Aplicaci\'on
  \begin{itemize}
    \item Generaci\'on de tr\'afico constante (\textit{CBR}).
    \item Programaci\'on del inicio y finalizaci\'on del tr\'afico.
  \end{itemize}
\end{enumerate}

\subsubsection*{Script 3}

\noindent Crea un simulador NS-2 \textbf{Simulator} y asigna colores para distinguir flujos de datos en NAM.
\begin{figure}[H]
  \begin{lstlisting}[frame=single, breaklines=true, basicstyle=\footnotesize\ttfamily, breakatwhitespace=false, 
    columns=flexible, tabsize=2, showstringspaces=false]
    set ns [new Simulator]
    $ns color 1 Blue
    $ns color 2 Red  
  \end{lstlisting}
\end{figure}

\noindent Abre archivos para registrar trazas generales \textbf{out.tr} y espec\'ificas para NAM \textbf{out.nam}.
\begin{figure}[H]
  \begin{lstlisting}[frame=single, breaklines=true, basicstyle=\footnotesize\ttfamily, breakatwhitespace=false, 
    columns=flexible, tabsize=2, showstringspaces=false]
    set file1 [open out.tr w]
    set winfile [open WinFile w]
    $ns trace-all $file1
    set file2 [open out.nam w]
    $ns namtrace-all $file2
  \end{lstlisting}
\end{figure}

\noindent Finaliza la simulaci\'on, cierra los archivos de traza y lanza NAM para visualizaci\'on.
\begin{figure}[H]
  \begin{lstlisting}[frame=single, breaklines=true, basicstyle=\footnotesize\ttfamily, breakatwhitespace=false, 
    columns=flexible, tabsize=2, showstringspaces=false]
    proc finish {} {
      global ns file1 file2
      $ns flush-trace
      close $file1
      close $file2
      exec nam out.nam &
      exit 0
    }
  \end{lstlisting}
\end{figure}

\noindent Crea nodos \textbf{n0} a \textbf{n5} y define enlaces entre ellos con las siguientes caracter\'isticas:
\begin{description}
  \item[Ancho de banda] : 2 Mbps
  \item[Retraso] : 10 ms
  \item[colas] : DropTail con tamaño de 20
\end{description}
\noindent Adem\'as orienta visualmente los enlaces en NAM. Relaciona la \textbf{Capa de Enlace de Datos} y la \textbf{Capa de Red}.
Ya que configura enlaces f\'isicos y colas adem\'as de modelar el retardo y ancho de banda.
\begin{figure}[H]
  \begin{lstlisting}[frame=single, breaklines=true, basicstyle=\footnotesize\ttfamily, breakatwhitespace=false, 
    columns=flexible, tabsize=2, showstringspaces=false]
    set n0 [$ns node]
    set n1 [$ns node]
    set n2 [$ns node]
    set n3 [$ns node]
    set n4 [$ns node]
    set n5 [$ns node]

    $ns duplex-link $n0 $n2 2Mb 10ms DropTail
    $ns duplex-link $n1 $n2 2Mb 10ms DropTail
    $ns simplex-link $n2 $n3 0.3Mb 100ms DropTail
    $ns simplex-link $n3 $n2 0.3Mb 100ms DropTail
    $ns duplex-link $n3 $n4 0.5Mb 40ms DropTail
    $ns duplex-link $n3 $n5 0.5Mb 30ms DropTail
  \end{lstlisting}
\end{figure}

\noindent Configura un agente TCP NewReno en el nodo \textbf{n0} y un receptor (TCPSink) en el nodo \textbf{n4}.
Ajusta par\'ametros como tamaño de ventana y tamaño de paquetes. Relaciona la \textbf{Capa de Transporte} con un protocolo
confiable (TCP).
\begin{figure}[H]
  \begin{lstlisting}[frame=single, breaklines=true, basicstyle=\footnotesize\ttfamily, breakatwhitespace=false, 
    columns=flexible, tabsize=2, showstringspaces=false]
    set tcp [new Agent/TCP/Newreno]
    $ns attach-agent $n0 $tcp
    set sink [new Agent/TCPSink/DelAck]
    $ns attach-agent $n4 $sink
    $ns connect $tcp $sink
    $tcp set fid_ 1
    $tcp set window_ 8000
    $tcp set packetSize_ 552
  \end{lstlisting}
\end{figure}

\noindent Configura una fuente FTP sobre TCP para generar tr\'afico. Relaciona la \textbf{Capa de Aplicaci\'on} emulando
una transferencia de archivos.
\begin{figure}[H]
  \begin{lstlisting}[frame=single, breaklines=true, basicstyle=\footnotesize\ttfamily, breakatwhitespace=false, 
    columns=flexible, tabsize=2, showstringspaces=false]
    set ftp [new Application/FTP]
    $ftp attach-agent $tcp
    $ftp set type_ FTP
  \end{lstlisting}
\end{figure}

\noindent Configura un agente UDP en el nodo \textbf{n1} y un receptor (Null) en el nodo \textbf{n5}.
Relaciona la \textbf{Capa de Transporte} con un protocolo no confiable (UDP).
\begin{figure}[H]
  \begin{lstlisting}[frame=single, breaklines=true, basicstyle=\footnotesize\ttfamily, breakatwhitespace=false, 
    columns=flexible, tabsize=2, showstringspaces=false]
    set udp [new Agent/UDP]
    $ns attach-agent $n1 $udp
    set null [new Agent/Null]
    $ns attach-agent $n5 $null
    $ns connect $udp $null
    $udp set fid_ 2
  \end{lstlisting}
\end{figure}

\noindent Configura una fuente CBR sobre UDP para generar tr\'afico constante. Relaciona la \textbf{Capa de Aplicaci\'on} simulando
tr\'afico constante en la red.
\begin{figure}[H]
  \begin{lstlisting}[frame=single, breaklines=true, basicstyle=\footnotesize\ttfamily, breakatwhitespace=false, 
    columns=flexible, tabsize=2, showstringspaces=false]
    set cbr [new Application/Traffic/CBR]
    $cbr attach-agent $udp
    $cbr set type_ CBR
    $cbr set packet_size_ 1000
    $cbr set rate_ 0.01mb
    $cbr set random_ false
  \end{lstlisting}
\end{figure}

\noindent Inicia y detiene las aplicaciones FTP y CBR en momentos espec\'ificos. Controla la \textbf{Capa de Aplicaci\'on}, definiendo
tiempos de inicio y fin.
\begin{figure}[H]
  \begin{lstlisting}[frame=single, breaklines=true, basicstyle=\footnotesize\ttfamily, breakatwhitespace=false, 
    columns=flexible, tabsize=2, showstringspaces=false]
    $ns at 0.1 "$cbr start"
    $ns at 1.0 "$ftp start"
    $ns at 124.0 "$ftp stop"
    $ns at 124.5 "$cbr stop"
  \end{lstlisting}
\end{figure}

\noindent Monitorea el tamaño de ventana de congesti\'on \textbf{cwnd} de TCP y registra los valores en un archivo
para an\'alisis posterior. Relacionado con la \textbf{Capa de Transporte}, ya que mide el comportamiento de TCP.
\begin{figure}[H]
  \begin{lstlisting}[frame=single, breaklines=true, basicstyle=\footnotesize\ttfamily, breakatwhitespace=false, 
    columns=flexible, tabsize=2, showstringspaces=false]
    proc plotWindow {tcpSource file} {
      global ns
      set time 0.1
      set now [$ns now]
      set cwnd [$tcpSource set cwnd_]
      puts $file "$now $cwnd"
      $ns at [expr $now+$time] "plotWindow $tcpSource $file"
    }
    $ns at 0.1 "plotWindow $tcp $winfile"
  \end{lstlisting}
\end{figure}


\subsection*{¿C\'omo podr\'ia usarse NS-2 para explicar la relaci\'on entre el retardo de propagaci\'on y el de transmisi\'on en un
enlace?¿Qu\'e parte de un script TCL deber\'ia modificar para ejemplificar esto?}
\noindent En NS-2 se puede explicar esta relaci\'on al configurar las propiedades del enlace entre nodos, espec\'ificamente
el ancho de banda y el retardo del enlace.
\begin{description}
  \item[Retarde de propagaci\'on] : Se configura directamente en la l\'inea donde se define el enlace entre nodos, como el tercer
  argumento. Repreenta el tiempo que tarda una señal en propagarse desde el origen al destino.
  \item[Retarde de transmisi\'on] : Depende del tamaño del paquete y del ancho de banda configurado en el enlace.
\end{description}

\noindent Modificando el \textbf{ancho de banda}, se puede observar c\'omo afecta el tiempo necesario para transmitir paquetes m\'asicas
grandes. Mientras que cambiando el \textbf{retardo del enlace}, se puede observar c\'omo afecta el tiempo total para que un paquete
llegue al destino, independientemente de su tamaño. 

\begin{figure}[H]
  \centering
  \begin{lstlisting}[frame=single, breaklines=true, basicstyle=\footnotesize\ttfamily, breakatwhitespace=false, 
    columns=flexible, tabsize=2, showstringspaces=false]
    $ns duplex-link $n0 $n1 1Mb 10ms DropTail
  \end{lstlisting}
  \caption{Modificaci\'on del script para ejemplificar la relaci\'on}
\end{figure}

\subsection*{Para el segundo ejemplo de la pr\'actica, modifique el c\'odigo para que el tamaño del paquete sea igual a 1,500 bytes.
¿Qu\'e diferencias observa en la animaci\'on con NAM?}
\noindent Cambiar el tamaño del paquete a 1,500 bytes afectar\'a el tiempo de transmisi\'on en el enlace. Si el tamaño del paquete aumenta, el 
tiempo de transmisi\'on tambi\'en aumentar\'a, lo que resultar\'a en una animaci\'on m\'as lenta en NAM.
\begin{itemize}
  \item Se observa que los paquetes tardan más tiempo en recorrer los enlaces, especialmente si hay varios paquetes en tránsito.
  \item El enlace puede congestionarse más r\'apido, lo que genera p\'erdida de paquetes en colas configuradas con DropTail.
\end{itemize}

\begin{figure}[H]
  \centering
  \begin{lstlisting}[frame=single, breaklines=true, basicstyle=\footnotesize\ttfamily, breakatwhitespace=false, 
    columns=flexible, tabsize=2, showstringspaces=false]
    $cbr0 set packetSize_ 1500
  \end{lstlisting}
  \caption{C\'odigo modificado}
\end{figure}


\subsection*{Investigue la disponibilidad de simuladores de red y realice una tabla con al menos cinco simuladores que compare sus
caracter\'isticas b\'asicas.}

\begin{table}[H]
  \begin{tabular}{@{}ccccc@{}}
  \toprule
  Simulador     & \begin{tabular}[c]{@{}c@{}}Lenguaje\\ principal\end{tabular} & \begin{tabular}[c]{@{}c@{}}Modelos\\ soportados\end{tabular}                 & Interfaz                                                     & Licencia                                                        \\ \midrule
  NS-2          & C++, Tcl                                                     & \begin{tabular}[c]{@{}c@{}}TCP, UDP, \\ Multicast, Routing\end{tabular}      & \begin{tabular}[c]{@{}c@{}}Basada en\\ Scritps\end{tabular}  & \begin{tabular}[c]{@{}c@{}}Open Source\\ GPL\end{tabular}       \\
  NS'3          & C++, Python                                                  & \begin{tabular}[c]{@{}c@{}}TCP, UDP, WiFi,\\ IPv4/IPv6, LTE\end{tabular}     & \begin{tabular}[c]{@{}c@{}}Basada en\\ Scripts\end{tabular}  & \begin{tabular}[c]{@{}c@{}}Open Source\\ GPLv2\end{tabular}     \\
  OMNeT++       & C++                                                          & \begin{tabular}[c]{@{}c@{}}Redes cableadas,\\ inálambricas, IoT\end{tabular} & \begin{tabular}[c]{@{}c@{}}Gráfica y\\ Scripts\end{tabular}  & \begin{tabular}[c]{@{}c@{}}Open Source\\ Academic\end{tabular}  \\
  Mininet       & Python                                                       & \begin{tabular}[c]{@{}c@{}}Redes definidas\\ por Software (SDN)\end{tabular} & \begin{tabular}[c]{@{}c@{}}Línea de \\ comandos\end{tabular} & \begin{tabular}[c]{@{}c@{}}Open Source\\ MIT\end{tabular}       \\
  Packet Tracer & Propietario                                                  & \begin{tabular}[c]{@{}c@{}}Modelos de Cisco\\ (RIP, OSPF, STP)\end{tabular}  & Gráfica                                                      & \begin{tabular}[c]{@{}c@{}}Propietario\\ Cisco\end{tabular}     \\
  GNS3          & Python                                                       & \begin{tabular}[c]{@{}c@{}}Redes virtualizadas\\ y físicas\end{tabular}      & Gráfica                                                      & \begin{tabular}[c]{@{}c@{}}Open Source\\ GNU GPLv3\end{tabular} \\ \bottomrule
  \end{tabular}
  \end{table}

\newpage
\subsubsection*{Detalles destacados}
\begin{description}
  \item[NS-2]  
  \begin{itemize}
    \item Enfoque: Simulación de protocolos de red
    \item Ventaja: Una gran cantidad de documentación y ejemplos debido a su uso histórico.  
    \item Desventaja: Es más antiguo y menos actualizado que NS-3.
  \end{itemize}
  
  \item[NS-3]   
  \begin{itemize}
    \item Enfoque: Simulación de redes modernas
    \item Ventaja: Mejor soporte para redes modernas como WiFi y LTE.
    \item Desventaja: Es más complejo que NS-2.
  \end{itemize}
  
  \item[OMNeT++]
  \begin{itemize}
    \item Enfoque: Modelado de redes y sistemas distribuidos
    \item Ventaja: Perfecto para simulaciones gráficas detalladas.
    \item Desventaja: No es tan popular para simulaciones prácticas como NS-3
  \end{itemize}

  \item[Mininet]
  \begin{itemize}
    \item Enfoque: Redes definidas por software (SDN)
    \item Ventaja: Ideal para probar redes definidas por software (SDN).
    \item Desventaja: No es tan adecuado para redes físicas tradicionales.
  \end{itemize}
  
  \item[Packet Tracer]
  \begin{itemize}
    \item Enfoque: Simulación de redes Cisco.
    \item Ventaja: Excelente herramienta educativa para estudiantes de redes.
    \item Desventaja: No es adecuado para simulaciones avanzadas fuera del ecosistema Cisco.
  \end{itemize}

  \item[GNS3]
  \begin{itemize}
    \item Enfoque: Redes reales con simulaciones de hardware.
    \item Ventaja: Soporta redes virtuales y físicas.
    \item Desventaja: Requiere más recursos de hardware que otros simuladores.
  \end{itemize}
\end{description}
