\section*{Cuestionario}

\subsection*{Describa qu\'e hace cada script ejecutado en la pr\'actica, ilustrando las partes en donde se configura cada una de las
capas del Modelo TCP/IP.}

\subsubsection*{Script 1}
\noindent Ya que este script no incluye nodos, enlaces, o tr\'afico de paquetes, vamos a describir cada parte del script.

\noindent Crea una instancia del simulador NS-2 y la asigna a la variable ns. Este objeto es el n\'ucleo de la simulaci\'on, permitiendo
condigurar nodos, tr\'afico, y otros elementos.
\begin{figure}[H]
  \centering
  \begin{lstlisting}[frame=single, breaklines=true, basicstyle=\footnotesize\ttfamily, breakatwhitespace=false, 
    columns=flexible, tabsize=2, showstringspaces=false]
    set ns [new Simulator]
  \end{lstlisting}
\end{figure}

Crea un archivo \textbf{out.nam} en modo escritura. Este archivo registrar\'a todos los eventos de simulaci\'on, como el env\'io/recepci\'on
de paquetes y otros eventos relevantes
\begin{figure}[H]
  \centering
  \begin{lstlisting}[frame=single, breaklines=true, basicstyle=\footnotesize\ttfamily, breakatwhitespace=false, 
    columns=flexible, tabsize=2, showstringspaces=false]
    set nf [open out.nam w]
  \end{lstlisting}
\end{figure}

Le indica al simulador que registre todos los eventos de simulaci\'on en el archivo \textbf{out.nam} previamente creado.
\begin{figure}[H]
  \centering
  \begin{lstlisting}[frame=single, breaklines=true, basicstyle=\footnotesize\ttfamily, breakatwhitespace=false, 
    columns=flexible, tabsize=2, showstringspaces=false]
    $ns namtrace-all $nf
  \end{lstlisting}
\end{figure}

\newpage

Define una funci\'on llamada \textbf{finish} que:
\begin{description}
  \item[global ns nf] : Declara las variables \textbf{ns} y \textbf{nf} como globales para usarlas dentro de la funci\'on.
  \item[\$ns flush-trace] : Asegura que toda la informaci\'on de traza se escriba en el archivo antes de cerrarlo.
  \item[close \$nf] : Cierra el archivo de traza \textbf{out.nam}
  \item[exec nam out.nam \&] : Ejecuta NAM y abre el archivo \textbf{out.nam} para visualizar la simulaci\'on.
  \item[exit 0] : Finaliza la ejecuci\'on del script de manera exitosa.      
\end{description}
\begin{figure}[H]
  \centering
  \begin{lstlisting}[frame=single, breaklines=true, basicstyle=\footnotesize\ttfamily, breakatwhitespace=false, 
    columns=flexible, tabsize=2, showstringspaces=false]
    proc finish {} {
      global ns nf
      $ns flush-trace
      close $nf
      exec nam out.nam &
      exit 0
    }
  \end{lstlisting}
\end{figure}

Programa la ejecuci\'on del procedimiento \textbf{finish} despu\'es de 5 segundos de tiempo simulado. Esto significa, que 
independientemente de la configuraci\'on adicional, la simulaci\'on terminar\'a a los 5 segundos.
\begin{figure}[H]
  \centering
  \begin{lstlisting}[frame=single, breaklines=true, basicstyle=\footnotesize\ttfamily, breakatwhitespace=false, 
    columns=flexible, tabsize=2, showstringspaces=false]
    $ns at 5.0 "finish"
  \end{lstlisting}
\end{figure}

Inicia la ejecuci\'on de la simulaci\'on. A partir de este punto, todos los eventos configurados en el simulador 
se procesan secuencialmente.
\begin{figure}[H]
  \centering
  \begin{lstlisting}[frame=single, breaklines=true, basicstyle=\footnotesize\ttfamily, breakatwhitespace=false, 
    columns=flexible, tabsize=2, showstringspaces=false]
    $ns run
  \end{lstlisting}
\end{figure}

\newpage

\subsubsection*{Script 2}

\noindent Como vimos en el punto anterior las siguientes l\'ineas del script son parte de la configuraci\'on general y la
creaci\'on del procedimiento de finalizaci\'on.
\begin{figure}[H]
  \centering
  \begin{lstlisting}[frame=single, breaklines=true, basicstyle=\footnotesize\ttfamily, breakatwhitespace=false, 
    columns=flexible, tabsize=2, showstringspaces=false]
    set ns [new Simulator]
    set nf [open out.nam w]
    $ns namtrace-all $nf
    proc finish {} {
      global ns nf
      $ns flush-trace
      close $nf
      exec nam out.nam &
      exit 0
    }
  \end{lstlisting}
\end{figure}

\noindent Crea dos nodos (\textit{n0 y n1}) que actuar\'an como origen y destino en la simulaci\'on. Esta parte del script corresponde a la
\textbf{Capa de Enlace de Datos}, ya que los nodos representan interfaces de red.
\begin{figure}[H]
  \centering
  \begin{lstlisting}[frame=single, breaklines=true, basicstyle=\footnotesize\ttfamily, breakatwhitespace=false, 
    columns=flexible, tabsize=2, showstringspaces=false]
    set n0 [$ns node]
    set n1 [$ns node]
  \end{lstlisting}
\end{figure}

\noindent Crea un enlace d\'uplex entre los nodos \textbf{n0} y \textbf{n1} con:
\begin{description}
  \item[Ancho de banda] : 1 Mbps
  \item[Retardo] : 10 ms
  \item[Cola] : DropTail (cola FIFO)   
\end{description}
La relaci\'on con el modelo TCP/IP es con la \textbf{Capa de Enlace de Datos} ya que representa el enlace f\'isico y la cola de paquetes.
As\'i como la \textbf{Capa de Red} ya que el retardo y el ancho de banda afectan la entrega de paquetes.
\begin{figure}[H]
  \centering
  \begin{lstlisting}[frame=single, breaklines=true, basicstyle=\footnotesize\ttfamily, breakatwhitespace=false, 
    columns=flexible, tabsize=2, showstringspaces=false]
    $ns duplex-link $n0 $n1 1Mb 10ms DropTail
  \end{lstlisting}
\end{figure}

\noindent Crea un agente UDP (\textit{udp0}) y lo asocia al nodo \textbf{n0}. Se relaciona con el modelo TCP/IP ya que representa la \textbf{Capa de Transporte}, 
ya que el protocolo UDP es un protocolo en esta capa.
\begin{figure}[H]
  \centering
  \begin{lstlisting}[frame=single, breaklines=true, basicstyle=\footnotesize\ttfamily, breakatwhitespace=false, 
    columns=flexible, tabsize=2, showstringspaces=false]
    set udp0 [new Agent/UDP]
    $ns attach-agent $n0 $udp0    
  \end{lstlisting}
\end{figure}

\newpage

\noindent Crea una  fuente de tr\'afico CBR (\textit{cbr0}) con:
\begin{description}
  \item[Tamaño de paquete] : 1000 bytes
  \item[Intervalo entre paquetes] : 0.005 segundos
  \item[Se conecta al agente UDP \textit{udp0}]  
\end{description}
Representa la \textbf{Capa de Aplicaci\'on}, ya que genera tr\'afico constante para enviar a trav\'es de la red.
\begin{figure}[H]
  \begin{lstlisting}[frame=single, breaklines=true, basicstyle=\footnotesize\ttfamily, breakatwhitespace=false, 
    columns=flexible, tabsize=2, showstringspaces=false]
    set cbr0 [new Application/Traffic/CBR]
    $cbr0 set packetSize_ 1000
    $cbr0 set interval_ 0.005
    $cbr0 attach-agent $udp0
  \end{lstlisting}
\end{figure}

\noindent Crea un agente Null (\textit{null0}) que act\'ua como receptor de tr\'afico en el nodo \textbf{n1}. Representa la \textbf{Capa de Transporte}, ya
que recibe y descarta paquetes sin procesarlos.
\begin{figure}[H]
  \begin{lstlisting}[frame=single, breaklines=true, basicstyle=\footnotesize\ttfamily, breakatwhitespace=false, 
    columns=flexible, tabsize=2, showstringspaces=false]
    set null0 [new Agent/Null]
    $ns attach-agent $n1 $null0    
  \end{lstlisting}
\end{figure}

\noindent Conecta el agente UDP (\textit{udp0}) en el nodo \textbf{n0} con el agente Null (\textit{null0}) en el nodo \textbf{n1}. Relaciona las 
\textbf{Capas de Transporte y Red}, al definir una ruta l\'ogica entre el origen y el destino.
\begin{figure}[H]
  \begin{lstlisting}[frame=single, breaklines=true, basicstyle=\footnotesize\ttfamily, breakatwhitespace=false, 
    columns=flexible, tabsize=2, showstringspaces=false]
    $ns connect $udp0 $null0
  \end{lstlisting}
\end{figure}

\noindent Programa el inicio del tr\'afico CBR a los 0.5 segundos y su detenci\'on a los 4.5 segundos de tiempo de simulaci\'on. Impacta la
\textbf{Capa de Aplicaci\'on}, ya que controla cu\'ando se env\'ian los datos.
\begin{figure}[H]
  \begin{lstlisting}[frame=single, breaklines=true, basicstyle=\footnotesize\ttfamily, breakatwhitespace=false, 
    columns=flexible, tabsize=2, showstringspaces=false]
    $ns at 0.5 "$cbr0 start"
    $ns at 4.5 "$cbr0 stop"    
  \end{lstlisting}
\end{figure}

\begin{enumerate}
  \item Capa de Enlace de Datos
  \begin{itemize}
    \item Creaci\'on de nodos.
    \item Configuraci\'on del enlace d\'uplex.
  \end{itemize}

  \item Capa de Red
  \begin{itemize}
    \item Configuraci\'on del enlace afecta el enrutamiento y entrega de paquetes.
  \end{itemize}

  \item Capa de Transporte
  \begin{itemize}
    \item Configuraci\'on del agente UDP y receptor Null.
  \end{itemize}

  \item Capa de Aplicaci\'on
  \begin{itemize}
    \item Generaci\'on de tr\'afico constante (\textit{CBR}).
    \item Programaci\'on del inicio y finalizaci\'on del tr\'afico.
  \end{itemize}
\end{enumerate}

\subsubsection*{Script 3}

Este script en Tcl configura una simulación de red usando NS-2, que incluye enlaces simplex y duplex, protocolos TCP y UDP, y aplicaciones FTP y CBR (Constant Bit Rate). Aquí describo cómo el script se relaciona con las capas del modelo TCP/IP:

---

### **Descripción General**

El script configura:
1. Enlaces entre nodos con diferentes características (ancho de banda, retraso).
2. Protocolos TCP y UDP con configuraciones específicas.
3. Aplicaciones FTP y CBR que generan tráfico de red.
4. Trazas para analizar el comportamiento de la red y su visualización en NAM.

---

### **Partes del Script y su Relación con el Modelo TCP/IP**

#### **1. Configuración General**
```tcl
set ns [new Simulator]
$ns color 1 Blue
$ns color 2 Red
```
- **Función:**
  Crea un simulador NS-2 (`Simulator`) y asigna colores para distinguir flujos de datos en NAM.
- **Relación con el Modelo TCP/IP:**
  No interactúa directamente con las capas, pero facilita la configuración y visualización.

---

#### **2. Archivos de Traza**
```tcl
set file1 [open out.tr w]
set winfile [open WinFile w]
$ns trace-all $file1
set file2 [open out.nam w]
$ns namtrace-all $file2
```
- **Función:**
  Abre archivos para registrar trazas generales (`out.tr`) y específicas para NAM (`out.nam`).
- **Relación con el Modelo TCP/IP:**
  Permite registrar eventos de todas las capas involucradas.

---

#### **3. Procedimiento `finish`**
```tcl
proc finish {} {
  global ns file1 file2
  $ns flush-trace
  close $file1
  close $file2
  exec nam out.nam &
  exit 0
}
```
- **Función:**
  Finaliza la simulación, cierra los archivos de traza y lanza NAM para visualización.
- **Relación con el Modelo TCP/IP:**
  No interactúa con las capas, pero organiza el cierre de la simulación.

---

#### **4. Configuración de Nodos y Enlaces (Capa de Enlace de Datos y Red)**
```tcl
set n0 [$ns node]
set n1 [$ns node]
# ...
$ns duplex-link $n0 $n2 2Mb 10ms DropTail
$ns simplex-link $n2 $n3 0.3Mb 100ms DropTail
$ns queue-limit $n2 $n3 20
```
- **Función:**
  - Crea nodos (`n0` a `n5`) y define enlaces entre ellos con:
    - **Ancho de banda** (e.g., `2Mb`).
    - **Retraso** (e.g., `10ms`).
    - **Colas** (e.g., `DropTail` con tamaño 20).
  - Orienta visualmente los enlaces en NAM.
- **Relación con el Modelo TCP/IP:**
  - **Capa de Enlace de Datos:** Configura enlaces físicos y colas.
  - **Capa de Red:** Modela el retardo y el ancho de banda.

---

#### **5. Configuración de TCP (Capa de Transporte)**
```tcl
set tcp [new Agent/TCP/Newreno]
$ns attach-agent $n0 $tcp
set sink [new Agent/TCPSink/DelAck]
$ns attach-agent $n4 $sink
$ns connect $tcp $sink
$tcp set fid_ 1
$tcp set window_ 8000
$tcp set packetSize_ 552
```
- **Función:**
  - Configura un agente TCP NewReno en `n0` y un receptor (TCPSink) en `n4`.
  - Ajusta parámetros como tamaño de ventana y tamaño de paquetes.
- **Relación con el Modelo TCP/IP:**
  Implementa la **Capa de Transporte** con un protocolo confiable (TCP).

---

#### **6. Configuración de FTP (Capa de Aplicación)**
```tcl
set ftp [new Application/FTP]
$ftp attach-agent $tcp
$ftp set type_ FTP
```
- **Función:**
  - Configura una aplicación FTP sobre TCP para generar tráfico.
- **Relación con el Modelo TCP/IP:**
  Implementa la **Capa de Aplicación**, emulando una transferencia de archivos.

---

#### **7. Configuración de UDP (Capa de Transporte)**
```tcl
set udp [new Agent/UDP]
$ns attach-agent $n1 $udp
set null [new Agent/Null]
$ns attach-agent $n5 $null
$ns connect $udp $null
$udp set fid_ 2
```
- **Función:**
  - Configura un agente UDP en `n1` y un receptor (Null) en `n5`.
- **Relación con el Modelo TCP/IP:**
  Implementa la **Capa de Transporte** con un protocolo no confiable (UDP).

---

#### **8. Configuración de CBR (Capa de Aplicación)**
```tcl
set cbr [new Application/Traffic/CBR]
$cbr attach-agent $udp
$cbr set type_ CBR
$cbr set packet_size_ 1000
$cbr set rate_ 0.01mb
$cbr set random_ false
```
- **Función:**
  - Configura una fuente CBR para generar tráfico constante sobre UDP.
- **Relación con el Modelo TCP/IP:**
  Implementa la **Capa de Aplicación**, simulando tráfico constante.

---

#### **9. Planificación de Eventos**
```tcl
$ns at 0.1 "$cbr start"
$ns at 1.0 "$ftp start"
$ns at 124.0 "$ftp stop"
$ns at 124.5 "$cbr stop"
```
- **Función:**
  - Inicia y detiene las aplicaciones FTP y CBR en momentos específicos.
- **Relación con el Modelo TCP/IP:**
  Controla la **Capa de Aplicación**, definiendo tiempos de inicio y fin.

---

#### **10. Monitoreo de TCP**
```tcl
proc plotWindow {tcpSource file} {
  global ns
  set time 0.1
  set now [$ns now]
  set cwnd [$tcpSource set cwnd_]
  puts $file "$now $cwnd"
  $ns at [expr $now+$time] "plotWindow $tcpSource $file"
}
$ns at 0.1 "plotWindow $tcp $winfile"
```
- **Función:**
  - Monitorea el tamaño de ventana de congestión (`cwnd`) de TCP.
  - Registra los valores en un archivo para análisis posterior.
- **Relación con el Modelo TCP/IP:**
  Relacionado con la **Capa de Transporte**, ya que mide el comportamiento de TCP.

---

### **Resumen de Capas del Modelo TCP/IP**
1. **Capa de Enlace de Datos:** Configuración de nodos, enlaces y colas.
2. **Capa de Red:** Parámetros de retardo y ancho de banda de los enlaces.
3. **Capa de Transporte:** Configuración de TCP y UDP.
4. **Capa de Aplicación:** Configuración de FTP y CBR.

El script modela la interacción entre múltiples nodos y protocolos, proporcionando una simulación detallada de red.

\subsection*{¿C\'omo podr\'ia usarse NS-2 para explicar la relaci\'on entre el retardo de propagaci\'on y el de transmisi\'on en un
enlace?¿Qu\'e parte de un script TCL deber\'ia modifical para ejemplificar esto?}

\subsection*{Para el segundo ejemplo de la pr\'actica, modifique el c\'odigo para que el tamaño del paquete sea igual a 1,500 bytes.
¿Qu\'e diferencias observa en la animaci\'on con NAM?}

\subsection*{Investigue la disponibilidad de simuladores de red y realice una tabla con al menos cinco simuladores que compare sus
caracter\'isticas b\'asicas.}

Aquí tienes una tabla que compara cinco simuladores de red ampliamente utilizados, con sus características clave:

| **Simulador**       | **Lenguaje principal** | **Modelos soportados**              | **Interfaz**         | **Escalabilidad**   | **Licencia**           | **Ventajas**                                             | **Desventajas**                                                |
|---------------------|-----------------------|-------------------------------------|----------------------|---------------------|-------------------------|---------------------------------------------------------|----------------------------------------------------------------|
| **NS-2**            | C++ y Tcl            | TCP, UDP, Multicast, Routing, etc. | Basada en scripts    | Moderada            | Open Source (GPL)       | Extensibilidad, biblioteca de modelos rica             | Complejidad en configuración y falta de mantenimiento activo. |
| **NS-3**            | C++ y Python         | TCP, UDP, IPv4/IPv6, WiFi, LTE, etc.| Basada en scripts    | Alta                | Open Source (GPLv2)     | Moderno, soporte IPv6 y simulación más precisa         | Mayor curva de aprendizaje para principiantes.                |
| **OMNeT++**         | C++                  | Redes cableadas, inalámbricas, IoT | Gráfica y scripts    | Alta                | Open Source (Academic)  | Interfaz gráfica, flexibilidad en simulaciones complejas | Más orientado a investigación académica.                      |
| **Mininet**         | Python               | Redes definidas por software (SDN) | Línea de comandos    | Alta (a nivel SDN)  | Open Source (MIT)       | Simulación rápida de redes SDN, fácil de usar          | Limitado para redes físicas no-SDN.                           |
| **Packet Tracer**   | Propietario          | Modelos de Cisco (RIP, OSPF, STP)  | Gráfica              | Baja/Moderada       | Propietario (Cisco)     | Fácil de usar, ideal para educación                    | Limitado a protocolos y equipos de Cisco.                     |
| **GNS3**            | Python               | Redes virtualizadas y físicas      | Gráfica              | Alta (con virtualización)| Open Source (GNU GPLv3) | Integración con dispositivos reales y virtualización   | Requiere hardware potente para simulaciones grandes.          |

---

### **Detalles destacados:**

1. **NS-2:**  
   - Ventaja: Una gran cantidad de documentación y ejemplos debido a su uso histórico.  
   - Desventaja: Es más antiguo y menos actualizado que NS-3.

2. **NS-3:**  
   - Ventaja: Mejor soporte para redes modernas como WiFi y LTE.  
   - Desventaja: Es más complejo que NS-2.

3. **OMNeT++:**  
   - Ventaja: Perfecto para simulaciones gráficas detalladas.  
   - Desventaja: No es tan popular para simulaciones prácticas como NS-3.

4. **Mininet:**  
   - Ventaja: Ideal para probar redes definidas por software (SDN).  
   - Desventaja: No es tan adecuado para redes físicas tradicionales.

5. **Packet Tracer:**  
   - Ventaja: Excelente herramienta educativa para estudiantes de redes.  
   - Desventaja: No es adecuado para simulaciones avanzadas fuera del ecosistema Cisco.

Espero que esta tabla te sea útil para seleccionar un simulador adecuado según tus necesidades. Si necesitas más información sobre alguno de ellos, ¡puedes pedírmelo!
