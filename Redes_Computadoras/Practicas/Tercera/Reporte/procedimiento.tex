\section*{Procedimiento}

\subsection*{Identificaci\'on de las trazas de audio}
\noindent Para empezar se descargan las trazas y descomprimirlas utilizando los siguientes comando en la terminal de Linux
\begin{figure}[H]
  \centering
  \begin{lstlisting}[frame=single, breaklines=true, basicstyle=\footnotesize\ttfamily, breakatwhitespace=false, 
      columns=flexible, tabsize=2, showstringspaces=false, language=bash] 
    
    wget http://victor.ramos.online.fr/practica1/traces/1.txt.gz
    
    gzip -dk 1.txt.gz

  \end{lstlisting}
  \caption{Comando para descargas y descomprimir las trazas}
  \label{fig:commandDownloadExtract}
\end{figure}

\noindent Para poder caracterizar el retardo de extremo a extremo, la Profra. Sue Moon propone añadir un elemento más a la
diferencia; dicho elemento es la mínima diferencia de retardo encontrada en toda la traza. Esto significa encontrar
\( t_{min}=min\{ t^r_i\}-t^t_i, \forall i\). Como las estampas de tiempo est\'an codificadas con el protocolo RTP y han sido
obtenidas con voz muestreada a 8,000 Hz, finalmente podremos observar el comportamiento del retardo de extremo a extremo
dentro del paquete \textit{i} con:
\begin{equation}
  d^i_{end-to-end}=\frac{t^r_i-t^t_i-t_{min}}{8000}[seg.]
\end{equation}

\noindent Para encontrar el \textit{tiempo de sesi\'on} para cada paquete, el an\'alisis es id\'entico al de una llamada
telef\'onica tradicional, el que llama paga. Entonces, el tiempo de sesi\'on estar\'a basado en el reloj del transmisor. Este
tiempo comenzar\'a a correr a partir de la primera estampara de tiempo \( t^t_i \) hasta la \'ultima \( t^t_N \) dado que
\textit{N} es el n\'umero total de paquetes en la sesi\'on. Entonces:
\begin{equation}
  t^i_{session}=\frac{t^t_i-t^t_i}{8000}[seg]
\end{equation}

\noindent Para analizar el retardo de extremo a extremo, entonces, se realizar\'an las gr\'aficas de las Ecuaciones (1) y (2)
para cada una de las seis trazas, a gran escala y a pequeña escala. La ecuaci\'on (1) es para el eje de las \textit{y} y (2)
para el eje de las \textit{x}.

\subsection*{Manipulaci\'on de trazas con AWK}
\noindent Para analizar algunas estad\'isticas b\'asicas de las sesiones de VoIP capturadas en las trazas, se relizaron los
siguientes scripts en AWK:
\begin{enumerate}
  \item Script para contar el n\'umero total de frases (\textit{talkspurts}) en una traza de retardo.
  \begin{figure}[H]
    \centering
    \begin{lstlisting}[frame=single, breaklines=true, basicstyle=\footnotesize\ttfamily, breakatwhitespace=false, 
        columns=flexible, tabsize=2, showstringspaces=false, language=AWK] 
      
        # Una frase terminal cuando se encuentra un periodo de silencio. Por lo tanto
        # podemos contar los periodos de silencio como separadores de frases
        BEGIN{
          talkspurts = 1
          input_file = ARGV[1] # Obtiene el nombre del archivo del parametro de la ejeucion
          gsub("../", "", input_file) # Elimina la parte inicial del parametro para tener nada mas el nombre del archivo
        }

        $1 == "!" { 
          talkspurts++
        }

        END{
          print "Archivo de entrada:", input_file
          print "Numero total de frases:", talkspurts
          print ""
        }

    \end{lstlisting}
    \label{fig:scriptTalksprut}
  \end{figure}

  \noindent A continuaci\'on se muestra el comando que se us\'o para la ejecuci\'on del script. Ya que ten\'ia que cumplir los 
  siguientes requisitos para mantener el orden de las carpetas:
  \begin{itemize}
    \item La ubicaci\'on del script no es la misma que las trazas.
    \item La ubicaci\'on del script deb\'ia ser una ruta establecida.
    \item Ya que el resultado para cada traza ser\'a una l\'inea, todas deben estar en el mismo archivo.
  \end{itemize}
  \begin{figure}[H]
    \centering
    \begin{lstlisting}[frame=single, breaklines=true, basicstyle=\footnotesize\ttfamily, breakatwhitespace=false, 
      columns=flexible, tabsize=2, showstringspaces=false, language=bash]

      awk -f ./script1.awk ../1.txt > ./output
      awk -f ./script1.awk ../6.txt >> ./output
      
    \end{lstlisting}
  \end{figure}

  \newpage
  \item Script para contar el n\'umero total de paquetes que llegaron al receptor.
  \begin{figure}[H]
    \centering
    \begin{lstlisting}[frame=single, breaklines=true, basicstyle=\footnotesize\ttfamily, breakatwhitespace=false, 
        columns=flexible, tabsize=2, showstringspaces=false, language=AWK] 
  
        # Cada paquete tiene una linea que comienza con "D".
        BEGIN{
            paquetes = 0
            input_file = ARGV[1] # Obtiene el nombre del archivo del parametro de la ejeucion
            gsub("../", "", input_file) # Elimina la parte inicial del parametro para tener nada mas el nombre del archivo  
        }

        $1 == "D"{
            paquetes++
        }

        END{
            print "Archivo de entrada:", input_file
            print "Numero total de paquetes recibidos:", paquetes
            print ""
        }

    \end{lstlisting}
    \label{fig:scriptPaquetesRecibidos}
  \end{figure}

  \noindent Para la ejecuci\'on de este script usaremos como base la forma vista en el punto anterior. 
  Cambiando el nombre del script de \textit{script1.awk} a \textit{script2.awk}

  \item Script para encontrar la diferencia m\'inima entre la estampa de tiempo de receptor menos de la del emisor
  \begin{figure}[H]
    \centering
    \begin{lstlisting}[frame=single, breaklines=true, basicstyle=\footnotesize\ttfamily, breakatwhitespace=false, 
        columns=flexible, tabsize=2, showstringspaces=false, language=AWK] 
  
        # Este script encuentra la diferencia minima entre la estampa del receptor y la del emisor.
        BEGIN{
            min_diff = "Inf"
            input_file = ARGV[1] # Obtiene el nombre del archivo del parametro de la ejeucion
            gsub("../", "", input_file) # Elimina la parte inicial del parametro para tener nada mas el nombre del archivo  
        }

        $1 == "D"{  # Solo se calcula la diferencia cuando hay un paquete recibido
            diff = $2 - $3 # Ya que solo es la diferencia no se toma encuenta que el numero que se obtiene puede ser negativo
            if (diff < min_diff){
                min_diff = diff
            }
        }

        END{
            print "Archivo de entrada:", input_file
            print "Diferencia minima:", min_diff   
            print ""
        }

    \end{lstlisting}
    \label{fig:scriptDiferencia}
  \end{figure}

  \noindent Para la ejecuci\'on de este script usaremos como base la forma vista en el punto anterior. 
  Cambiando el nombre del script de \textit{script2.awk} a \textit{script3.awk}

  \item Script que recibe una traza como entrada y entrega en un archivo dos columnas: tiempo de la sesi\'on en segundos, 
  y retardo de extremo a extremo a extremo en segundo.
  \begin{figure}[H]
    \centering
    \begin{lstlisting}[frame=single, breaklines=true, basicstyle=\footnotesize\ttfamily, breakatwhitespace=false, 
        columns=flexible, tabsize=2, showstringspaces=false, language=AWK] 
  
        # Este script genera un archivo con dos columnas: tiempo de la sesion y retardo extremo a extremo.
        BEGIN{
            #Llama al script que se encarga de obtener la diferencia minima
            temp_file = "output.tmp"
            input_file = ARGV[1]
            command = "awk -f ../Script_3/script3.awk "input_file " > " temp_file
            system(command)
        
            # Procesa la salida del archivo temporal
            while ((getline line < temp_file) > 0) {
                # Procesa cada linea de salida del segundo script
                if (line ~ /Diferencia minima:/) {
                    split(line, fields, ":")
                    tmin = fields[2]
                }
            }
        
            # Limpia el archivo temporal
            system("rm -f " temp_file)
        
            t1_flag = 0
        }
        
        $1 == "D"{
            if (t1_flag == 0){
                t1 = $3
                t1_flag = 1
            }
        
            session_time = ($3 - t1) / 8000
            end_to_end_delay = ($2 - $3 - tmin) / 8000
            print session_time, end_to_end_delay
        }
              
    \end{lstlisting}
    \label{fig:scriptTiempoSesionRetardoExtremoExtremo}
  \end{figure}

  \noindent A continuaci\'on se muestra el comando que se us\'o para la ejecuci\'on del script. Ya que se tiene que generar
  un archivo para cada una de las trazas 
  \begin{figure}[H]
    \centering
    \begin{lstlisting}[frame=single, breaklines=true, basicstyle=\footnotesize\ttfamily, breakatwhitespace=false, 
      columns=flexible, tabsize=2, showstringspaces=false, language=bash]

      awk -f ./script1.awk ../1.txt > ./output
      
    \end{lstlisting}
  \end{figure}

  \newpage
  \item Script que calula el retardo promedio de extremo a extremo en una sesi\'on
  \begin{figure}[H]
    \centering
    \begin{lstlisting}[frame=single, breaklines=true, basicstyle=\footnotesize\ttfamily, breakatwhitespace=false, 
        columns=flexible, tabsize=2, showstringspaces=false, language=AWK] 
  
        # Este script calcula el retardo promedio extremo a extremo.
        BEGIN{
            tmin = "Inf" # Inicializa con el valor infinito
            total_delay = 0
            input_file = ARGV[1]
            gsub("../", "", input_file)
        }
        
        $1 == "D"{
        
            if ($2-$3 < tmin) {
                tmin = $2-$3
            }
            
            delay = ($2 - $3 - tmin) / 8000
            total_delay += delay
            count++
        }
        
        END{
            print "Archivo de entrada:", input_file
            if (count > 0) {
                print "Retardo promedio extremo a extremo:", total_delay / count
            } else {
                print "No se encontraron datos para calcular el retardo promedio."
            }
            print ""
        }
                
    \end{lstlisting}
    \label{fig:scriptRetardoExtremoExtremoSesion}
  \end{figure}
  \noindent Para la ejecuci\'on de este script usaremos como base la forma vista en el primer punto. 
  Cambiando el nombre del script de \textit{script1.awk} a \textit{script5.awk}
\end{enumerate}