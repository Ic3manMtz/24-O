\section*{Procedimiento}

\subsection*{Identificaci\'on de las trazas de audio}
\noindent Para empezar se descargan las trazas y descomprimirlas utilizando los siguientes comando en la terminal de Linux
\begin{figure}[H]
  \centering
  \begin{lstlisting}[frame=single, breaklines=true, basicstyle=\footnotesize\ttfamily, breakatwhitespace=false, 
      columns=flexible, tabsize=2, showstringspaces=false, language=bash] 
    
    wget http://victor.ramos.online.fr/practica1/traces/1.txt.gz
    
    gzip -dk 1.txt.gz

  \end{lstlisting}
  \caption{Comando para descargas y descomprimir las trazas}
  \label{fig:commandDownloadExtract}
\end{figure}

\noindent Para poder caracterizar el retardo de extremo a extremo, la Profra. Sue Moon propone añadir un elemento más a la
diferencia; dicho elemento es la mínima diferencia de retardo encontrada en toda la traza. Esto significa encontrar
\( t_{min}=min\{ t^r_i\}-t^t_i, \forall i\). Como las estampas de tiempo est\'an codificadas con el protocolo RTP y han sido
obtenidas con voz muestreada a 8,000 Hz, finalmente podremos observar el comportamiento del retardo de extremo a extremo
dentro del paquete \textit{i} con:
\begin{equation}
  d^i_{end-to-end}=\frac{t^r_i-t^t_i-t_{min}}{8000}[seg.]
\end{equation}

\noindent Para encontrar el \textit{tiempo de sesi\'on} para cada paquete, el an\'alisis es id\'entico al de una llamada
telef\'onica tradicional, el que llama paga. Entonces, el tiempo de sesi\'on estar\'a basado en el reloj del transmisor. Este
tiempo comenzar\'a a correr a partir de la primera estampara de tiempo \( t^t_i \) hasta la \'ultima \( t^t_N \) dado que
\textit{N} es el n\'umero total de paquetes en la sesi\'on. Entonces:
\begin{equation}
  t^i_{session}=\frac{t^t_i-t^t_i}{8000}[seg]
\end{equation}

\noindent Para analizar el retardo de extremo a extremo, entonces, se realizar\'an las gr\'aficas de las Ecuaciones (1) y (2)
para cada una de las seis trazas, a gran escala y a pequeña escala. La ecuaci\'on (1) es para el eje de las \textit{y} y (2)
para el eje de las \textit{x}.

\subsection*{Manipulaci\'on de trazas con AWK}
\noindent Para analizar algunas estad\'isticas b\'asicas de las sesiones de VoIP capturadas en las trazas, se relizaron los
siguientes scripts en AWK:
\begin{enumerate}
  \item Script para contar el n\'umero total de frases (\textit{talkspurts}) en una traza de retardo.
  \begin{figure}[H]
    \centering
    \begin{lstlisting}[frame=single, breaklines=true, basicstyle=\footnotesize\ttfamily, breakatwhitespace=false, 
        columns=flexible, tabsize=2, showstringspaces=false, language=AWK] 
      
    \end{lstlisting}
    \label{fig:scriptTalksprut}
  \end{figure}

  \item Script para contar el n\'umero total de paquetes que llegaron al receptor.
  \begin{figure}[H]
    \centering
    \begin{lstlisting}[frame=single, breaklines=true, basicstyle=\footnotesize\ttfamily, breakatwhitespace=false, 
        columns=flexible, tabsize=2, showstringspaces=false, language=AWK] 
  
    \end{lstlisting}
    \label{fig:scriptPaquetesRecibidos}
  \end{figure}

  \item Script para encontrar la diferencia m\'inima entre la estampa de tiempo de receptor menos de la del emisor
  \begin{figure}[H]
    \centering
    \begin{lstlisting}[frame=single, breaklines=true, basicstyle=\footnotesize\ttfamily, breakatwhitespace=false, 
        columns=flexible, tabsize=2, showstringspaces=false, language=AWK] 
  
    \end{lstlisting}
    \label{fig:scriptDiferencia}
  \end{figure}

  \item Script que recibe una traza como entrada y entrega en un archivo dos columnas: tiempo de la sesi\'on en segundos, 
  y retardo de extremo a extremo a extremo en segundo.
  \begin{figure}[H]
    \centering
    \begin{lstlisting}[frame=single, breaklines=true, basicstyle=\footnotesize\ttfamily, breakatwhitespace=false, 
        columns=flexible, tabsize=2, showstringspaces=false, language=AWK] 
  
    \end{lstlisting}
    \label{fig:scriptTiempoSesionRetardoExtremoExtremo}
  \end{figure}

  \item Script que calula el retardo promedio de extremo a extremo en una sesi\'on
  \begin{figure}[H]
    \centering
    \begin{lstlisting}[frame=single, breaklines=true, basicstyle=\footnotesize\ttfamily, breakatwhitespace=false, 
        columns=flexible, tabsize=2, showstringspaces=false, language=AWK] 
  
    \end{lstlisting}
    \label{fig:scriptRetardoExtremoExtremoSesion}
  \end{figure}
\end{enumerate}