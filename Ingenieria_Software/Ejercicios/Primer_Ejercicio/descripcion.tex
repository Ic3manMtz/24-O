%-----------Descripción del problema 

\section*{Descripción del problema}

Dentro del mundo de los gimnasios podemos encontrar dos roles: los entrenadores o asesores, y los que acuden al gimnasio en búsqueda de un cambio físico. Es muy común que los segundos busquen entrenadores para obtener rutinas y consejos nutricionales para lograr los resultados que quieren de manera más efectiva. 

Sin embargo, la mayoría de los entrenadores trabajan independientemente del gimnasio por lo que no cuentan con un sistema para llevar la gestión de sus clientes. Por lo general su contacto es por WhatsApp; las rutinas y los planes alimenticios suelen ser creadas en las notas del teléfono del entrenador, y en algunos casos cuentan con plantillas en Word que mandan a sus clientes. 

Esto provoca que los entrenadores no tengan una gestión eficiente de sus clientes. Así como una personalización de las rutinas y planes alimenticios asignados. Otra problemática es que los entrenadores no tienen forma de llevar registro del progreso de sus clientes a menos que se tengan un contacto constante con ellos.