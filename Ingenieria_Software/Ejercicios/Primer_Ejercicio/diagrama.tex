%-----------Diagrama de la arquitectura 

\section*{Diagrama de la arquitectura}

Para este desarrollo se usó la arquitectura de tres capas (\textit{3-layer}) es una estructura de diseño de software que organiza una aplicación en tres niveles distintos, cada uno con responsabilidades específicas. Aquí te explico cada una de las capas:

\begin{enumerate}
    \item \textbf{Capa de Presentación} - Es la interfaz de usuario de la aplicación. Aquí es donde los usuarios interactúan con el sistema.
    \item \textbf{Capa de Aplicación} - Conocida como capa lógica o de negocio, es donde se procesan los datos y se aplican las reglas de negocio.
    \item \textbf{Capa de Datos} - Se encarga del almacenamiento y gestión de datos. Aquí es donde se accede a las bases de datos y se realizan las operaciones CRUD (\textit{Create, Read, Update, Delete}).
\end{enumerate}

\newpage
\textbf{Beneficios de la Arquitectura de Tres Capas}

\begin{enumerate}
    \item \textbf{Separación de responsabilidades}
    \item \textbf{Desarrollo en paralelo}
    \item \textbf{Reutilización de componentes}
    \item \textbf{Escabilidad}
\end{enumerate}
