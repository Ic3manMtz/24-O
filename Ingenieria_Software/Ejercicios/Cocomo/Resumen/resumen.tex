\section*{Resumen}

\subsection*{Introducci\'on}
\noindent El modelo COCOMO (\textit{Constructive Cost Model}) fue desarrollado por Barry Boehm en 1981 y se utiliza para estimar los costos
de proyectos de software. Se basa en una familia de modelos de algoritmos de costos y ha evolucionado para reflejar nuevas tecnolog\'ias y
pr\'acticas de la Ingenier\'ia de Software.

\subsection*{Tipos de modelos COCOMO}
\begin{description}
  \item[COCOMO 81] Se divide en tres submodelos:
  \begin{itemize}
    \item Modelo b\'asico: Usa una funci\'on simple basada en el tamaño del software (\textit{en miles de l\'ineas de c\'odigo}) para calcular
    el esfuerzo.
    \item Modelo intermedio: Agrega 15 controladores de costo que permiten ajustar la estimaci\'on.
    \item Modelo detallado: Eval\'ua los controladores de costo en cada fase del ciclo de vida del proyecto.
  \end{itemize} 
  
  \item[COCOMO II] Se desarrolla para abordar enfoques modernos de desarrollo de software. Introduce factores de escala y controladores de
  costo adicionales. Proporciona estimaciones probabil\'isticas en lugar de valores puntuales. Utiliza tres modelos:
  \begin{itemize}
    \item Composici\'on de aplicaciones
    \item Diseño temprano
    \item Post-arquitectura
  \end{itemize}
\end{description}

\subsection*{Modos de desarrollo}
\begin{description}
  \item[Org\'anico] Proyetos pequeños, con baja innovaci\'on y sin restricciones.
  \item[Semi.disjunto] Proyectos de tamaño medio con restricciones parciales.
  \item[Embebido] Proyectos grandes con alta innovaci\'on y restricciones importantes.  
\end{description}

\subsection*{Controladores de costo}
\noindent Se eval\'uan 15 factores clave, entre ellos la confiabilidad (\textit{RELY}), la complejidad del producto (\textit{CPLX}), la
experiencia en el lenguaje de programaci\'on (\textit{LEXP}), y la disponibilidad de herramientas de software (\textit{TOOL}), cada uno
con sus respectivos multiplicadores de esfuerzo.

\subsection*{Ecuaciones b\'asicas}

$$ Esfuerzo (MM): MM=a*(KDSI)^b $$
$$ Tiempo de desarrollo (TDEV): TDEV=2.5*(MM)^c $$

\noindent Los coeficientes a, b y c var\'ian seg\'un el modo de desarrollo.

\subsection*{Ventajas de COCOMO}
\noindent Transparente y verificable además de facilitar la compresi\'on de los factores que afectan los costos.

\subsection*{Desventajas de COCOMO}
\noindent Dificultad para estimar con precisi\'on el tamañoo del software antes de comenzar. Alta dependencia de la clasificaci\'on correcta
del modo de desarrollo. Vulnerabilidad ante la falta de datos hist\'oricos precisos.

\subsection*{Diferencias clave entre COCOMO 81 y COCOMO II}
\noindent COCOMO II usa KSLOC (\textit{l\'ineas de c\'odigo l\'ogico}) en lugar de KDSI. COCOMO II proporciona estimaciones de rangos de
probabilidad. COCOMO II  se ajusta mejor a proyctos con reutilizaci\'on de software y reingenier\'ia.